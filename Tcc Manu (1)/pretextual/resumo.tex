\begin{resumo}
%Elemento obrigatório, constituído de uma sequência de frases concisas e objetivas, em forma de texto.  Deve apresentar os objetivos, métodos empregados, resultados e conclusões.  O resumo deve ser redigido em parágrafo único, conter no máximo 500 palavras e ser seguido dos termos representativos do conteúdo do trabalho (palavras-chave).

A Internet vem se tornando cada vez mais popular e indispensável para a maioria da população. Hoje a Internet é uma das maiores fontes de informações, utilizada para a realização de diversas atividades do dia a dia, tais como: transações bancárias, compras \textit{online}, entre outras. Porém  é necessário garantir a segurança dos dados que trafegam tanto na Internet como em quaisquer outras redes de computadores. Para que a segurança dos dados ocorram, vulnerabilidades existentes em dispositivos  conectados à rede devem ser encontradas e corrigidas, mas a detecção de vulnerabilidades nem sempre é uma tarefa simples, pois na maioria das vezes, vulnerabilidades estão contidas em softwares de terceiros, ou até mesmo nas configurações das redes, realizadas pelos próprios administradores e podem acabar passando despercebidas. Entretanto, existem programas que são utilizados para analisar os dispositivos conectados na rede em busca de informações, tais programas são chamados de \textit{scanners}, e podem auxiliar no gerenciamento de redes de computadores. No entanto, quando as redes possuem centenas de dispositivos conectados, os \textit{scanners} tendem a retornar grandes quantidades de informações, demandando um grande tempo para que os administradores analisem as informações, testem as vulnerabilidades encontradas e caso sejam confirmadas, corrijam tais vulnerabilidades.
Dessa maneira, este trabalho propõe uma arquitetura capaz de auxiliar na detecção e visualização de vulnerabilidades em redes de computadores. Sensores serão utilizados com o objetivo de obter a maior quantidade de informações disponíveis a respeito dos dispositivos que serão analisados. Tais informações serão utilizadas para criar perfis para os dispositivos da rede. Estes perfis serão armazenadas, criando uma base de dados contendo todas as informações obtidas. Essas informações serão utilizadas para auxiliar o usuário na administração da rede, as informações podem ser mostrada de maneira gráfica, através de uma interface web, facilitando o entendimento dos resultados obtidos. Além disso, para validar a arquitetura proposta será implementada uma ferramenta, na qual será utilizada dois \textit{scanners} para obter informações a respeito da rede analisadas. As informações obtidas serão utilizadas para auxiliar no gerenciamento e manutenção da rede monitorada. 
É esperado que a arquitetura consiga lidar com grandes quantidades de informações obtidas a partir da análise da rede, detectar vulnerabilidades recentes que possam causar ameaças à rede monitorada, além de gerar visualizações gráficas que auxiliem os usuários no gerenciamento e manutenção das redes de computadores.




% TODO: se possível, escreva um resumo estruturado. Para TCC 1, o resumo estruturado teria os seguintes elementos:
% \textbf{Contexto:} \\
% \textbf{Objetivo:} \\
% \textbf{Método:} \\
% \textbf{Resultados esperados:} 
% ou, para TCC 2:
% \textbf{Contexto:} \\
% \textbf{Objetivo:} \\
% \textbf{Método:} \\
% \textbf{Resultados:} \\
% \textbf{Conclusões:}

% Palavras-chaves, separadas por ponto (tente não definir mais do que cinco)
\palavraschaves{Vulnerabilidades, Redes, Segurança, Cibersegurança.}
\end{resumo}